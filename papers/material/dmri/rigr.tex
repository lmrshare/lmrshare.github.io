\documentclass[12pt,a4paper]{article}

\usepackage{CJKutf8}
\usepackage[unicode,pdftex]{hyperref}
\usepackage{amsmath,amssymb,amsfonts}
\usepackage{graphicx}
\usepackage{color}
\usepackage[framed,numbered,autolinebreaks,useliterate]{mcode}
\usepackage{geometry}
\geometry{a4paper,textwidth=17cm,textheight=27cm}

\setlength{\parindent}{0pt}
\setlength{\parskip}{18pt}

\begin{document}

\begin{CJK}{UTF8}{songti}

\title{RIGR}
\author{MingJian Hong \\ \texttt{hongmingjian@gmail.com}}
\date{September 24, 2011} % Activate to display a given date or no date (if empty),
         % otherwise the current date is printed
\maketitle

\textbf{Theory}

RIGR is a model-based method for dynamic MRI. In principle, RIGR uses the correlation of the data points within the k-space to reconstruct the image. Specifically, RIGR assumes that the object image can be represented by a generalized series (GS) with some proper basis derived from the Fourier basis. The coefficients of the series are obtained by solving a set of Toeplitz systems.

Here, we assume that the first frame of k-space $F_0$ is fully sampled in a grid of [0,$N$-1]x[0, $M$-1], while the k-space of subsequent frames $\hat{F}$ is under-sampled in $N_u$x[0, $M$-1] with $N_u$ is a subset of [0,$N$-1].

If the k-space of each frame is fully sampled, then the image $f(i, k)$ can be reconstructed by inverse DFT as follows.
\begin{equation}
f(i, k)=\frac{1}{N}\sum_{n=0}^{N-1}exp(2 \pi j \frac{ni}{N})\frac{1}{M}\sum_{m=0}^{M-1}F(n,m)exp(2 \pi j \frac{mk}{M})
\end{equation}
Now, considering that only rows in $N_u$ are available, we represent the image with a GS as follows.
\begin{equation}
\hat{f}(i,k)=\frac{1}{N}S(i,k)\sum_{t \in N_u} exp(2 \pi j\frac{ti}{N}) c_{t,k}
\label{gs}
\end{equation}
where $c_{t,k}$ is the coefficients of the GS and $S(i,k)$ absorbs available a prior information. By calculating the DFT of the $\hat{f}(i, k)$, we have
\begin{equation}
\hat{F}(n, m)=\sum_{i=0}^{N-1}\sum_{k=0}^{M-1}\hat{f}(i,k)exp(-2 \pi j(\frac{ni}{N}+\frac{mk}{M}))
\label{dft}
\end{equation}

Inserting the (\ref{gs}) into (\ref{dft}), we have
\begin{eqnarray*}
\hat{F}(n, m) & = & \frac{1}{N}\sum_{i=0}^{N-1}\sum_{k=0}^{M-1} S(i,k)\sum_{t \in N_u} exp(2 \pi j \frac{ti}{N}) c_{t,k} exp(-2 \pi j(\frac{ni}{N}+\frac{mk}{M})) \\
              & = & \frac{1}{N}\sum_{k=0}^{M-1}exp(-2 \pi j\frac{mk}{M})\sum_{t \in N_u}c_{t,k} (\sum_{i=0}^{N-1}S(i,k)exp(-2 \pi j\frac{n-t}{N}i))
\end{eqnarray*}
with $n \in N_u, m=[0, M-1]$.

That means
\begin{equation}
\frac{1}{N} \sum_{t \in N_u}c_{t,k} (\sum_{i=0}^{N-1}S(i,k)exp(-2 \pi j\frac{n-t}{N}i))=\frac{1}{M} \sum_{m=0}^{M-1}\hat{F}(n,m)exp(2 \pi j \frac{mk}{M})
\label{toeplitz}
\end{equation}
with $n \in N_u, k=[0, M-1]$. For each fixed $k$, equation (\ref{toeplitz})  is a Toeplitz system and can be sovled to get $c_{t,k}(t \in N_u)$, which is used to reconstruct the object image by eq.(\ref{gs}).

Usually, the $S(i, k)$ is set to $|I_0(i,k)|$ for every $i=[0,N-1]$ and
$k=[0,M-1]$, where $I_0(i,k)$ is the reference image.



\medskip

\textbf{C code for RIGR}

\url{http://mri.beckman.uiuc.edu/software.html}

\textbf{Matlab code for RIGR}

%1) \mcode{for i=1:3, disp('cool'); end;}

%\begin{lstlisting}
%	if i >= 5           % literate programming replacement
%		disp('cool');  % comment with some §\mcommentfont\LaTeX in it: $\mcommentfont\pi x^2$§
%	end
%	really really long really really long really really long really really long really really long line % blaaaaaaaa
%\end{lstlisting}

%\lstinputlisting[firstline=6, lastline=15]{/SOME/PATH/FILENAME.M}

\textbf{References}

[1] Z.-P. Liang and P. C. Lauterbur, “A generalized series approach to MR spectroscopic imaging”, Medical Imaging, IEEE Transactions on, vol. 10, no. 2, pp. 132-137, 1991.

[2] Z. P. Liang and P. C. Lauterbur, “An efficient method for dynamic magnetic resonance imaging”, IEEE Transactions on Medical Imaging, vol. 13, no. 4, pp. 677-686, 1994.

[3] E. L. Piccolomini, F. Zama, G. Zanghirati, and A. Formiconi, “Regularization methods in dynamic MRI”, Appl. Math. Comput., vol. 132, no. 2-3, pp. 325–339, Nov 2002.

[4] G. Landi, E. L. Piccolomini, and F. Zama, “A total variation-based reconstruction method for dynamic MRI”, Computational and Mathematical Methods in Medicine, vol. 9, no. 1, pp. 69–80, 2008.

%\bibliographystyle{amsplain}
%\bibliography{chemo}

\end{CJK}
\end{document}
